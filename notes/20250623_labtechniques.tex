\documentclass[letterpaper, 12pt]{article}
\usepackage{graphicx} % Required for inserting images
\usepackage{textcomp}
\usepackage{fullpage}
\usepackage{amsmath}
\usepackage{xcolor}
\usepackage{float}
\usepackage{geometry}
\usepackage{biblatex}
\geometry{margin=1in}
\usepackage{enumitem}
\usepackage{microtype}
\usepackage{gensymb}
\usepackage{parskip}
\usepackage{tikz}
\usepackage{caption}
\usepackage{cancel}
\usepackage{nicefrac}


\usepackage{hyperref}
\hypersetup{
    colorlinks=true,        % Enable colored links
    linkcolor=teal,         % Set color for internal links
    citecolor=teal,         % Set color for citations
    filecolor=teal,         % Set color for file links
    urlcolor=teal           % Set color for URLs
}

\usepackage[version=4]{mhchem}

\title{Techniques in protein biochemistry and thermodynamics}
\author{BIOS 10016}
\date{23 June 2025}

\begin{document}

\maketitle

\section*{Objectives}

\begin{itemize}
\item Know all definitions
\item Describe the importance of assays and differentiate between direct and indirect methods
\item Calculate specific activity, percent yield and purification level
\item Describe how assays and PAGE can be used to assess protein purity.
\item Describe the principles of PAGE and SDS-PAGE
\item Describe the theory, principles, conditions required and the results of the different purification procedures discussed:
\begin{itemize}
\item Size-exclusion chromatography
\item Ion exchange chromatography
\item Affinity purification
\end{itemize}
\item \textbf{EXAM I ENDS HERE}
\item Know all definitions
\item Describe the first two laws of thermodynamics
\item Assess if a reaction is endothermic, exothermic or isothermic
\item Describe the standard state and biological standard state conditions
\item Assess if and describe why a reaction is spontaneous or not
\item Assess if the entropy of a system increases or decreases
\item Differentiate between $\Delta$G$\degree$ and $\Delta$G$\degree$’
\item Calculate the equilibrium constant from $\Delta$G$\degree$ and vice versa
\item Calculate $\Delta$G from initial concentrations of substrates and products
\end{itemize}

\newpage

\section*{Laboratory techniques}

\subsection*{Protein purification}
Requires:

\begin{itemize}
\item A source (natural or genetically engineered)
\item A method to assess purity (activity assays, gel-based)
\end{itemize}

\subsubsection*{Direct activity assay}

Test used to measure the enzymatic activity of a specific enzyme by directly detecting the change in a substrate or the formation of a product as the reaction occurs.

\begin{description}
\item [activity] $\displaystyle \frac{\text{amount}}{\text{time}}$
\item [standard enzyme activity, U] $\displaystyle \frac{\mu}{\text{min}}$
\item [specific activity] $\displaystyle \frac{\text{enzyme activity (Units)}}{\text{mg of total protein}}$, a measure of purity (as mg of total protein decreases, activity increases, considering the amount of protein of interest remains the same)

\item {Percent yield} (how much activity is left after performing a certain step relative to original sample):

\begin{equation}
\text{Percent yield} = \frac{\text{remaining total activity}}{\text{initial total activity}} \times 100\%
\end{equation}

\item {Purification level} (how much purer the sample is after a certain step relative to original sample):

\begin{equation}
\text{Purification level} = \frac{\text{specific activity}}{\text{initial specific activity}} \times 100\%
\end{equation}

\end{description}

\subsubsection*{\textit{P}oly\textit{A}crylamide \textit{G}el \textit{E}lectrophoresis (PAGE)}

\begin{itemize}

\item Matrix is \textbf{polyacrylamide} that has holes and pores, changing how quickly the material can pass through in response to electric field. Bigger molecules move much slower
\item Separates proteins based on \textbf{charge and size}
\item \textbf{Cathode} (negative) attracts \textbf{positive} ions
\item \textbf{Anode} (positive) attracts \textbf{negative} ions
\item Proteins migrate towards electrode of opposite charge

\end{itemize}

\subsubsection*{\textit{S}odium \textit{D}odecyl \textit{S}ulfate-PAGE (SDS-PAGE)}

\begin{itemize}
\item SDS (negatively charged detergent) binds tightly to proteins (1.4 g SSDS/1 g protein)
\item All proteins become very negatively charged and separation depends only on size
\item 3$\degree$ and 4$\degree$ structure disrupted (denatured)
\item Often reducing conditions used: 
\begin{itemize}
\item DTT
\item $\beta$-Mercaptoethanol ($\beta$-ME)
\end{itemize}
\item $\implies$ disulfide bonds broken
\item Protein mobility $\displaystyle \propto \frac{1}{\log \text{(molecular weight)}}$
\item Smears indicate impurity; isolated proteins indicate purity
\end{itemize}

\subsection*{Chromatography}

\subsubsection*{Gel filtration/Size exclusion column chromatography}

\begin{itemize}
\item Separated based on size
\item Presence of a gel matrix with holes and gel beads attached
\item Proteins elute (come out) in order of decreasing molecular weight, shown by a peak in the graph
\item Larger molecules elute first (excluded from pores) because they interact less with the gel beads
\item Smaller molecules elute later (enter pores) because they interact more with the gel beads
\item Time of elution $\displaystyle \propto \frac{1}{\log \text{(molecular weight)}}$
\end{itemize}

\subsubsection*{Ion exchange chromatography}

\paragraph{Anion exchange chromatography} The \textbf{resin} is \textbf{positively charged}, the \textbf{protein} to be isolated is \textbf{negatively charged}.

\paragraph{Cation exchange chromatography} The \textbf{resin} is \textbf{negatively charged}, the \textbf{protein} to be isolated is \textbf{positively charged}.

\begin{itemize}
\item pH $>$ pI $\implies$ negative proteins
\item pH $<$ pI $\implies$ positive proteins
\item More charges $\to$ higher affinity
\item Once charged molecules are stuck to the resin, elution is done by changing pH or salt concentration in the buffer
\item Four steps:
\begin{itemize}
\item Equilibration
\item Sample application and wash
\item Elution
\item Regeneration
\end{itemize}
\end{itemize}

\subsection*{Affinity purification}

A \textbf{ligand} is used to isolate the protein (can be a substrate, inhibitor, or antibody that specifically binds to the target protein)

Hexahistidine tag: HHHHH - N or C terminus

binds \ce{Ni2+} or \ce{Co2+}

\begin{enumerate}
\item Column has \ce{Ni2+} and \ce{Co2+} attached. pH should be around 8.0 so that histidine is not positively charged.
\item Wash the protein-bound column to get rid of undesired proteins
\item Elute the protein by lowering the pH (protonating the histidine) or adding imidazole to disrupt electrostatic interaction
\item Imidazole binds, protein comes off
\end{enumerate}

\vspace{5em}

\begin{center}
\large \textbf{End of Exam I}
\end{center}

\newpage

\section*{Thermodynamics}

\subsection*{Key terms}

\begin{description}
\item [energy] is the capacity to do work or to transfer heat
\item [work] is organized motion that results in a specific physical change through the displacement or movement of an object
\item [heat] is energy transferred as the result of a temperature difference
\item [enthalpy (H)] total heat content of the system or surroundings
\item [entropy] a measure of the number of microstates in a system or the surroundings
\item [$\Delta$ (delta)] means ``change in''
\item [Gibbs free energy (G)] a measure of spontaneity; $\Delta G = \Delta H - T \Delta S$
\end{description}

\subsection*{The laws of thermodynamics}

\subsubsection*{First law}

\textbf{Energy cannot be created nor destroyed, but it can be changed from one form to another}

$$ \Delta H_\text{system} = - \Delta H_\text{surroundings} $$

Enthalpy (H) is related to internal energy of the system (U). In biological systems, $\Delta H = \Delta U$

\begin{description}
\item [exothermic] system releases heat to the surroundings $\Delta H_\text{system} < 0$
\item [endothermic] system absorbs heat from the surroundings $\Delta H_\text{system} > 0$
\item [isothermic] no exchange of heat with the surroundings $\Delta H_\text{system} = 0$
\end{description}

\subsubsection*{Second law}

\textbf{All spontaneous processes result in an overall increase of entropy in the universe}

$$ \Delta S_\text{system} + \Delta S_\text{surroundings} > 0$$

\textbf{Spontaneous} reactions occur automatically if left alone (no intervention required) and are energetically favourable

$$\ce{C6H12O6_{(s)} + 6O2_{(g)}} \to \ce{6CO2_{(g)} + 6H2O_{(g)}} \quad \Delta H = -2800 \mathrm{kJ/mol}$$

Entropy is a measure of \textbf{randomness}, and every spontaneous process results in a positive entropy change in the universe. How does this reaction result in increased entropy?

\begin{enumerate}
\item Number of molecules increases (from 7 to 12)
\item Heat released
\item Solid to gas
\end{enumerate}

\subsection*{How living systems counteract entropy}

\begin{itemize}
\item Living systems are very ordered!
\item To live we must increase the disorder of the surroundings
\item Ingest complex molecules and expel small molecules and heat
\item Organisms with $\Delta S_\text{univ} = 0$ are dead
\item THE SUN MAINTAINS ORDER! pop off queen
\end{itemize}

\subsection*{Describing spontaneous processes}

\subsubsection*{Gibbs free energy}

The entropic and enthalpic components of a reaction can be combined into a general description of spontaneity.

\begin{equation}
\Delta G_\text{system} = \Delta H - T \Delta S
\end{equation}

conceptually:

$$ \text{free energy of the system} = \text{entropy of surroundings} - \text{temperature} \times \text{entropy of the system} $$

\begin{description}
\item [exergonic] spontaneous, negative $\Delta G$
\item [endergonic] not spontaneous, positive $\Delta G$
\item [equilibrium] at $\Delta G = 0$
\end{description}

Standard free energy change ($\Delta$G$\degree$) tells us the direction of a reaction at \textbf{standard state}:

\begin{itemize}
\item 1M initial concentration
\item pressure = 1 atm
\item temperature can vary, generally 0$\degree$C or 25$\degree$C
\end{itemize}

In biological systems, we make an exception for [\ce{H+}]

\begin{itemize}
\item [\ce{H+}] = 1M, meaning the pH is 0
\item Under biological standard conditions, pH = 7, [\ce{H+}] = $1 \times 10^7$
\item $\Delta$G$\degree$' = \textbf{biological standard free energy change}
\end{itemize}

\subsubsection*{Calculating the $\Delta G$ away from equilibrium}

$$ \text{aA} + \text{bB} \rightleftharpoons \text{cC} + \text{dD} $$

If $\Delta G < 0$, the forward reaction is spontaneous. \\
If $\Delta G = 0$, the reaction is at equilibrium. \\
If $\Delta G < 0$, the reverse reaction is spontaneous.

Spontaneity depends on the reactant and product concentrations. You can change the temperature and concentration to favor certain reactions (forward and back reaction).

\textbf{When doing problems with temperature, always convert into Kelvin! ($\degree$C + 273)}

\subsection*{Enzymes}

Combustion of glucose

$$\ce{C6H12O6_{(s)} + 6O2_{(g)}} \to \ce{6CO2_{(g)} + 6H2O_{(g)}}$$

\textbf{Activation barrier} must first be overcome before a reaction can happen (sugar will not be combusting on its own)

Enzymes...

\begin{itemize}
\item are catalyst proteins (speed up a reaction)
\item end in -ase
\item catalyze 1000s of specific reactions in the human body
\item will always return to original conformation
\item accelerate reactions by lowering activation energy
\item are low in concentration relative to the substrates
\item bind substrates with high specificity
\item are highly regulated by inhibitors or modifications
\item do not change the K$_\text{eq}$ and $\Delta$G$\degree$ for the reaction; only affects the barrier
\item function depends on protein structure
\end{itemize}

\end{document}