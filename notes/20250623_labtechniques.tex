\documentclass[letterpaper, 12pt]{article}
\usepackage{graphicx} % Required for inserting images
\usepackage{textcomp}
\usepackage{fullpage}
\usepackage{amsmath}
\usepackage{xcolor}
\usepackage{float}
\usepackage{geometry}
\usepackage{biblatex}
\geometry{margin=1in}
\usepackage{enumitem}
\usepackage{microtype}
\usepackage{gensymb}
\usepackage{parskip}
\usepackage{tikz}
\usepackage{caption}
\usepackage{cancel}
\usepackage{nicefrac}


\usepackage{hyperref}
\hypersetup{
    colorlinks=true,        % Enable colored links
    linkcolor=teal,         % Set color for internal links
    citecolor=teal,         % Set color for citations
    filecolor=teal,         % Set color for file links
    urlcolor=teal           % Set color for URLs
}

\usepackage[version=4]{mhchem}

\title{Techniques in protein biochemistry}
\author{BIOS 10016}
\date{23 June 2025}

\begin{document}

\maketitle

\section*{Objectives}

\begin{itemize}
\item Know all definitions
\item Describe the importance of assays and differentiate between direct and indirect methods
\item Calculate specific activity, percent yield and purification level
\item Describe how assays and PAGE can be used to assess protein purity.
\item Describe the principles of PAGE and SDS-PAGE
\item Describe the theory, principles, conditions required and the results of the different purification procedures discussed:
\begin{itemize}
\item Size-exclusion chromatography
\item Ion exchange chromatography
\item Affinity Purification
\end{itemize}
\end{itemize}

\newpage

\section*{Laboratory techniques}

\subsection*{Protein purification}
Requires:

\begin{itemize}
\item A source (natural or genetically engineered)
\item A method to assess purity (activity assays, gel-based)
\end{itemize}

\subsubsection*{Direct activity assay}

Test used to measure the enzymatic activity of a specific enzyme by directly detecting the change in a substrate or the formation of a product as the reaction occurs.

\begin{description}
\item [activity] $\displaystyle \frac{\text{amount}}{\text{time}}$
\item [standard enzyme activity, U] $\displaystyle \frac{\mu}{\text{min}}$
\item [specific activity] $\displaystyle \frac{\text{enzyme activity (Units)}}{\text{mg of total protein}}$, a measure of purity (as mg of total protein decreases, activity increases, considering the amount of protein of interest remains the same)

\item {Percent yield} (how much activity is left after performing a certain step relative to original sample):

\begin{equation}
\text{Percent yield} = \frac{\text{remaining total activity}}{\text{initial total activity}} \times 100\%
\end{equation}

\item {Purification level} (how much purer the sample is after a certain step relative to original sample):

\begin{equation}
\text{Purification level} = \frac{\text{specific activity}}{\text{initial specific activity}} \times 100\%
\end{equation}

\end{description}

\subsubsection*{\textit{P}oly\textit{A}crylamide \textit{G}el \textit{E}lectrophoresis (PAGE)}

\begin{itemize}

\item Matrix is \textbf{polyacrylamide} that has holes and pores, changing how quickly the material can pass through in response to electric field. Bigger molecules move much slower
\item Separates proteins based on \textbf{charge and size}
\item \textbf{Cathode} (negative) attracts \textbf{positive} ions
\item \textbf{Anode} (positive) attracts \textbf{negative} ions
\item Proteins migrate towards electrode of opposite charge

\end{itemize}

\subsubsection*{\textit{S}odium \textit{D}odecyl \textit{S}ulfate-PAGE (SDS-PAGE)}

\begin{itemize}
\item SDS (negatively charged detergent) binds tightly to proteins (1.4 g SSDS/1 g protein)
\item All proteins become very negatively charged and separation depends only on size
\item 3$\degree$ and 4$\degree$ structure disrupted (denatured)
\item Often reducing conditions used: 
\begin{itemize}
\item DTT
\item $\beta$-Mercaptoethanol ($\beta$-ME)
\end{itemize}
\item $\implies$ disulfide bonds broken
\item Protein mobility $\displaystyle \propto \frac{1}{\log \text{(molecular weight)}}$
\item Smears indicate impurity; isolated proteins indicate purity
\end{itemize}

\subsection*{Chromatography}

\subsubsection*{Gel filtration/Size exclusion column chromatography}

\begin{itemize}
\item Separated based on size
\item Presence of a gel matrix with holes and gel beads attached
\item Proteins elute (come out) in order of decreasing molecular weight, shown by a peak in the graph
\item Larger molecules elute first (excluded from pores) because they interact less with the gel beads
\item Smaller molecules elute later (enter pores) because they interact more with the gel beads
\item Time of elution $\displaystyle \propto \frac{1}{\log \text{(molecular weight)}}$
\end{itemize}

\subsubsection*{Ion exchange chromatography}

\paragraph{Anion exchange chromatography} The \textbf{resin} is \textbf{positively charged}, the \textbf{protein} to be isolated is \textbf{negatively charged}.

\paragraph{Cation exchange chromatography} The \textbf{resin} is \textbf{negatively charged}, the \textbf{protein} to be isolated is \textbf{positively charged}.

\begin{itemize}
\item pH $>$ pI $\implies$ negative proteins
\item pH $<$ pI $\implies$ positive proteins
\item More charges $\to$ higher affinity
\item Once charged molecules are stuck to the resin, elution is done by changing pH or salt concentration in the buffer
\end{itemize}

\end{document}