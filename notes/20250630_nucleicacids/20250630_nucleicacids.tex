\documentclass[letterpaper, 12pt]{article}
\usepackage{graphicx} % Required for inserting images
\usepackage{textcomp}
\usepackage{fullpage}
\usepackage{amsmath}
\usepackage{xcolor}
\usepackage{float}
\usepackage{geometry}
\usepackage{biblatex}
\geometry{margin=1in}
\usepackage{enumitem}
\usepackage{microtype}
\usepackage{gensymb}
\usepackage{parskip}
\usepackage{tikz}
\usepackage{caption}
\usepackage{cancel}
\usepackage{nicefrac}


\usepackage{hyperref}
\hypersetup{
  colorlinks=true,        % Enable colored links
  linkcolor=teal,         % Set color for internal links
  citecolor=teal,         % Set color for citations
  filecolor=teal,         % Set color for file links
  urlcolor=teal           % Set color for URLs
}

\usepackage[version=4]{mhchem}

\title{Nucleic acids}
\author{BIOS 10016}
\date{30 Jun 2025}

\begin{document}

\maketitle

\section*{Objectives}

\begin{itemize}
\item Know all definitions associated with nucleic acids.
\item Draw and name the nucleosides, nucleotides and Watson-Crick base pairs associated with DNA and RNA.
\begin{itemize}
\item Nucleotides: adenine, guanine, thymine, cytosine, uracil
\item Nucleosides: adenosine, guanosine, thymidine, cytidine, uridine
\item Adenine $\to$ thymine/uracil (2 H-bonds), cytosine $\to$ guanine (3 H-bonds)
\end{itemize}
\item Know the numbering schemes for the components of nucleotides.
\begin{itemize}
\item Pyrimidines: start with sugar nitrogen, number around the ring in the direction of the closest nitrogen
\item Purines: start with nitrogen diagonally across sugar nitrogen, number around the ring in the direction of the closest nitrogen, jump to closest nitrogen on other ring, repeat
\end{itemize}
\item Describe and explain the absorbance properties of nitrogenous bases. (absorb UV light at 260 nm)
\item Describe the chemical and structural similarities and differences between DNA and RNA.
\begin{itemize}
\item DNA:
\begin{itemize}
\item Double-stranded
\item Deoxyribose (no OH) sugar
\item Thymine as a nitrogenous base
\item Stable, less reactive
\item Stores genetic information
\end{itemize}
\item RNA:
\begin{itemize}
\item Single-stranded (mostly)
\item Ribose (OH) sugar
\item Uracil as a nitrogenous base
\item Less stable, more reactive (enzymatic activity)
\item Involved in protein synthesis and regulation
\end{itemize}
\end{itemize}
\item Use Chargaff’s rules to predict the composition of DNA molecules. (\% nucleotide = \% complementary nucleotide)
\item Describe the different double helical structures that DNA and RNA can adopt. (B-DNA, A-DNA, Z-DNA, mRNA, tRNA, rRNA)
\item Write out the complementary strand based upon Watson-Crick base pairing.
\item Predict the stability of double stranded DNA sequences. (C to G is more stable, higher percentage of C-G bonds increases stability)
\item Identify major and minor grooves in B-DNA.
\item Describe how proteins interact with DNA, and why the major groove is often bound. (major groove is wider and deeper, allowing for more interactions with amino acid side chains. Positive amino side chains interact with the negatively charged phosphate backbone)
\item Describe the features/forces that stabilize nucleic acid structure. (LDFs stabilize overall structure, hydrogen bonds hold strands together)
\item Describe the phenomenon and role of supercoils, and the supercoil forms that exist.
\begin{itemize}
\item Positive supercoiling: overwinding, left-handed supercoil
\item Negative supercoiling: underwinding, right-handed supercoil
\end{itemize}
\item Describe the higher order structures that DNA adopts in cells. 
\item Describe the roles of messenger, transfer, ribosomal and other non-coding RNAs.
\item Describe the features and exceptions to the central Dogma. (DNA $\to$ RNA $\to$ protein through transcription and subsequently translation, but also RNA $\to$ DNA in retroviruses, and RNA can have catalytic roles)
\item Describe the concepts behind DNA replication:
\begin{itemize}
\item what is the meaning of semi-conservative? (one original strand, one daughter strand)
\item the stages involved in the process
\begin{itemize}
\item Initiation: helicase unwinds the protein, single-stranded binding proteins stabilize the unwound DNA, primase lays down RNA primer
\item Elongation: DNA polymerase synthesizes new DNA strand, topoisomerases relieve supercoiling
\item Termination: ligase seals gaps, telomerase extends telomeres, Tus protein terminates replication in prokaryotes
\end{itemize}
\item main protein players in prokaryotic and eukaryotic systems and
the roles that they play in the process
\item mechanism by which nucleotides are added to growing strand (5' continuously, 3' added in Okasaki fragments)
\end{itemize}
\end{itemize}

\newpage

\section*{Nucleic acids, RNA, DNA}

\subsection*{The building blocks of nucleic acids}

\subsubsection*{Nucleotides}
Nitrogenous bases with pentose sugar ($\beta$-D-ribofuranose or $\beta$-D-2'\footnote{The ' is added to differentiate between the sugar carbon numbering and the nucleotide numbering.}-deoxyribofuranose) with one or more phosphate groups ($\alpha$, $\beta$, $\gamma$). Nitrogenous bases can be \textbf{purines} with two rings, 4 nitrogens (adenine, guanine) or \textbf{pyrimidines} with one ring (cytosine, uracil, thymine). Nitrogenous bases are heterocycles, aromatic, and flat and absorb UV light (260 nm).

\subsubsection*{Nucleosides}
Nitrogenous bases with pentose sugar (ribose or deoxyribose) with no phosphate groups.

\subsubsection*{Numbering positions on nitrogenous bases}
\textbf{Pyrimidines}: start with the nitrogen where the sugar is attached, count in a circle toward the next nitrogen.

\textbf{Purines}: start with the position opposite the sugar nitrogen, then jump to the next closest nitrogen.

\subsubsection*{Nitrogenous base pairings}
adenine $\to$ thymine (in DNA), 2 hydrogen bonds \\
guanine $\to$ cytosine, 3 hydrogen bonds \\
adenine $\to$ uracil (in RNA), 2 hydrogen bonds

\subsubsection*{Nucleotide nomenclature}

\begin{table}[H]
\begin{tabular}{|c|c|c|}
\hline
\textbf{Nitrogenous base} & \textbf{Ribose nucleoside} & \textbf{Within a nucleic acid} \\\hline
Adenine & Adenosine & Adenylate \\\hline
Guanine & Guanosine & Guanylate \\\hline
Cytosine & Cytidine & Cytidylate \\\hline
Uracil & Uridine & Uridylate \\\hline
Thymine & Thymidine & Thymidylate \\\hline
\end{tabular}
\end{table}

\subsection*{Structure of DNA}

\subsubsection*{Nucleic acid orientation}
\begin{itemize}
\item Linear or circular, depending on organism
\item DNA - deoxyribonucleotide polymer: dAMP, dCMP, dGMP and TMP
\item Linked by phosphodiester bonds between the 5' phosphate of one nucleotide and the 3' hydroxyl of the next (3' $\to$ 5' linkage)
\item RNA - ribonucleotide polymer: AMP, CMP, GMP and UMP
\item Conventionally, named and drawn in the 5' to 3' direction (example: 5' ATGC 3')
\item Double helix structure of B-DNA was published in 1953 by James Watson and Francis Crick (but it was all because of Rosalind Franklin's X-ray diffraction data!)
\item Nobel prize in 1962 along with Maurice Wilkins
\item \textbf{Anti-parallel, right-handed helix}
\item Two grooves: major (wider, deeper) and minor (shallower, smaller)
\item 10.5 base pairs per turn
\item Watson-Crick base pairing (cytosine to guanine, thymine or uracil to adenine)
\end{itemize}

\subsubsection*{Chargaff's rule for double-stranded DNA}
\%C = \%G and \%A = \%T. G to C is more stable because of 3 hydrogen bonds

\subsubsection*{Base pair stacking}
\begin{itemize}
\item Base pairs stack on each other like books on a shelf
\item \textbf{Stabilized by London interactions between pi orbitals} ($\pi$-$\pi$ interaction)
\item Edges of bases define grooves
\end{itemize}

\subsubsection*{DNA conformations}
B-DNA ist he most common form adopted, but other forms exist:
\begin{itemize}
\item A-DNA (dehydrated B-DNA)
\begin{itemize}
\item Right-handed helix
\item Observed in RNA, dehydrated DNA, and DNA-RNA hybrid structures
\item 11 base pairs per turn
\item Pitch: 25.3 Å
\end{itemize}
\item Z-DNA
\begin{itemize}
\item Zigzag conformation
\item Left-handed helix
\item Common for alternating purine/pyrimidine sequences
\item Arises from \textbf{torsional strain} imposed by supercoiling
\item 12 base pairs per turn
\item Pitch: 45.6 Å
\end{itemize}
\end{itemize}

\subsection*{Amino acid interactions}
\begin{itemize}
\item Positively charged amino acids (R, H, K) form salt bridges with \ce{PO4-}, elevating protein pI
\item Amino acid residues ``read'' the DNA sequence by recognising H-bond acceptors (A) and donors (D) in the major groove
\begin{itemize}
\item Adenine: \ce{-NH2}, \ce{N7}
\item Thymine: \ce{O4}
\item Guanine: \ce{O6}, \ce{N7}
\item Cytosine: \ce{-NH2}
\end{itemize}
\end{itemize}

\subsection*{Structure stability}

\begin{itemize}
\item Hydrogen bonds within base pairs
\item Hydration (phosphate group and ribose OH are hydrophilic)
\item Ionic (phosphate backbone is rich in negative charges, high energy state that can be stabilized by Mg)
\item Hydrophobic interactions and base stacking (bases are aromatic - hydrophobic, London force interactions between $\pi$-$\pi$ orbitals result in stacking)
\end{itemize}

\subsection*{Supercoiling of DNA}

\begin{itemize}
\item Positive and negative supercoiling
\item Imposed strain causes supercoiling
\item Results from attempted unwinding of DNA
\item Energy is stored in the supercoil as torque
\item A right-handed coil
\item The energy helps in the separation of the strands for transcription and replication of DNA
\end{itemize}

\subsubsection*{Positive supercoiling}
Results from overwinding, a left-handed supercoil

\subsubsection*{Negative supercoiling}
Results from underwinding, a right-handed supercoil

\subsection*{Higher order structures of DNA}

\begin{itemize}
\item \textbf{Nucleosomes}: a complex between DNA and histone proteins
\item ``Beads on a string'' structure
\item Positively charged amino acids needed on histone proteins
\end{itemize}

\subsection*{RNA - Ribonucleic acids}
\begin{itemize}
\item A versatile molecule:
\begin{itemize}
\item Protein synthesis
\item Structural roles
\item Catalytic roles
\end{itemize}
\item Differs from DNA:
\begin{itemize}
\item Ribose sugar instead of deoxyribose
\item Uracil nitrogenous base replaces thymine
\item Predominantly exists in single-stranded form
\item Adjacent bases adopt 2$\degree$ structures: hairpins, A-form helices, internal loops, stem loops
\item Adopts 3$\degree$ and 4$\degree$ structures, very protein-like
\item Structures are also stabilized by unique base pairs/triplets
\end{itemize}
\item Modification of bases is common
\end{itemize}

\subsubsection*{Types of RNA}
\begin{description}
\item [messenger RNA, mRNA] carries genetic information from DNA through to protein synthesis
\item [transfer RNA, tRNA] molecules transport amino acids to the ribosomes for polypeptide assembly (15\% of cellular RNA)
\item [ribosomal RNA, rRNA] the functional component oand the framework for the ribosome
\item [ribozyme, catalytic RNA] formation of peptide bonds, associated proteins only play structural roles
\end{description}

\subsection*{Processes involving DNA}

\subsubsection*{Central Dogma of genetic information transfer}

\textbf{DNA transcripted into RNA translated into proteins (and DNA can replicate itself)}

\begin{itemize}
\item The traditional central dogma is overly simplified since information also flows in the other direction
\begin{itemize}
\item Proteins are involved in regulation
\item RNA molecules involved at all steps
\item Viruses have RNA genomes
\item \textbf{RNA world}: the hypothesis that RNA was the first self-replicating molecule
\item Like proteins, RNA molecules can perform many functional roles
\end{itemize}
\end{itemize}

\subsubsection*{DNA replication}

\begin{itemize}
\item Process by which DNA is duplicated
\item Similar in all organisms
\item \textbf{Semi-conservative}: new DNA molecules contain one parent and one daughter strand (original strand separates into 2, old strand + new strand)
\end{itemize}

\textbf{Steps:}

\begin{enumerate}
\item Initiation:
\begin{itemize}
\item Helicase (important to rip strands apart)
\item Single-stranded DNA binding proteins
\item Primase (allows recognition by polymerase, laying the foundation)
\end{itemize}
\item Elongation
\begin{itemize}
\item Polymerases
\item Topoisomerases (release supercoiling so processes continue)
\end{itemize}
\item Termination
\begin{itemize}
\item Ligase
\item Telomerase
\item Tus protein
\end{itemize}
\end{enumerate}

\subsubsection*{DNA replication sites}
\begin{itemize}
\item In prokaryotes: 1 site: oriC
\item In eukaryotes: multiple origin sites
\item Recognised by helicases that destabilise the double-stranded DNA duplex, requires ATP (energy)
\item Very energy-intensive process
\item Leads to the formation of \textbf{replication bubbles} from origins of replication
\item \textbf{Single-stranded DNA binding (SSB) proteins} stabilize single stranded DNA state
\item \textbf{Primase} ``primes'' the open DNA for replication
\item Recognises the exposed bases in the replication fork
\item Synthesises a short primer sequence on the DNA template (RNA in prokaryotes, RNA-DNA in eukaryotes)
\end{itemize}

\subsubsection*{DNA polymerase}
\begin{itemize}
\item Main polymerases: DNA polymerase III (prokayreotes) and DNA polymerases $\delta$ and $\varepsilon$ (eukaryotes)
\item Large multi-enzyme complex
\item Recognises the RNA-DNA duplex in the replication bubble
\item Highly processive, not released until completed replication of DNA
\item $\beta$-sliding clamp (encircles DNA and enhances processivity)
\item Requires \ce{Mg2+}, DNA template, primers with a free 3' OH group, and deoxyNTPs (dATP, dCTP, dGTP and TTP)
\end{itemize}

\subsubsection*{5' to 3' synthesis of DNA}
\begin{itemize}
\item DNA template read in the 3' $\to$ 5' direction
\item Parental DNA opens
\item 5' is the leading strand (continuous), 3' is the lagging strand (discontinuous, happens in \textbf{Okazaki fragments})
\item DNA made in the 5' $\to$ 3' direction
\end{itemize}

\end{document}