\documentclass[letterpaper, 12pt]{article}
\usepackage{graphicx} % Required for inserting images
\usepackage{textcomp}
\usepackage{fullpage}
\usepackage{amsmath}
\usepackage{xcolor}
\usepackage{float}
\usepackage{geometry}
\usepackage{biblatex}
\geometry{margin=1in}
\usepackage{enumitem}
\usepackage{hyperref}
\usepackage{microtype}
\usepackage{gensymb}
\usepackage{parskip}
\usepackage{tikz}
\usepackage{caption}
\usepackage{cancel}
\usepackage{nicefrac}
\hypersetup{
    colorlinks=true,        % Enable colored links
    linkcolor=teal,         % Set color for internal links
    citecolor=teal,         % Set color for citations
    filecolor=teal,         % Set color for file links
    urlcolor=teal           % Set color for URLs
}

\usepackage[version=4]{mhchem}

\title{Lab safety orientation}
\author{BIOS 1006}
\date{16 June 2025}

\begin{document}

\maketitle

\section*{General safety}

\subsection*{Rights and responsibilities}

\subsubsection*{Rights of the researcher}

\begin{itemize}
\item To understand hazards
\item To work in a safe environment
\item To relevant safety training
\item To medical consultation
\item To Personal Protective Equipment (PPE)
\item To file complaint
\end{itemize}

\subsubsection*{Responsibilities of the researcher}

\begin{itemize}
\item Following all safety rules
\item Obtaining all required safety trainings
\item Maintaining a safe work environment 
\item Notifying supervisor of unsafe work conditions or suspicious activity
\item Reporting all injuries and accidents
\end{itemize}

\subsection*{Common hazards}

\begin{itemize}
\item Slips, trips, and falls
\item Electrical
\item High temperature 
\item Sharps
\item Biohazards
\item Chemicals
\end{itemize}

\paragraph{Slips, trips, and falls} Comprise the majority of work-related incidents. Can be prevented with good housekeeping, proper storage, cleaning spills immediately, and not climbing on benchtops or chairs.

\paragraph{High temperature} Including Bunsen burners and hot plates. Hot and cold items look alike. Do not leave unattended.

\paragraph{Sharps} Different types of sharps can be found in a lab, such as needles. Dispose properly and do not recap needles.

\paragraph{Biohazards} Hazards derived from living things. This can include microbial pathogens (bacteria, viruses, fungi, etc.), material derived from humans (blood, tissues, cells, etc.), and recombinant DNA (CRISPR, GFP, antibiotic-resistant plasmids, etc.).

\subsection*{Safety equipment}
\begin{itemize}
\item Hand wash sink
\item Eye wash station
\item Safety shower
\item First-aid kit
\item Spill kit
\item Fire extinguishers
\end{itemize}

\section*{Biosafety}

\subsection*{Risk groups for biologicals (RGs)}

\subsubsection*{RG1}

\begin{itemize}
\item Not associated with disease in healthy adults
\item Many are beneficial (probiotics, microbiome) -- Food fermentation (bread, cheese, etc)
\item Cells from plants and animals (not human)
\end{itemize}

\subsubsection*{RG2}

\begin{itemize}
\item Cause diseases in healthy adults, but usually not serious/fatal and can be treated
\item Bacteria, viruses such as \textit{Salmonella}, pathogenic \textit{E. coli}
\item Any human-derived material
\item Culturing unknown samples from the environment
\end{itemize}

\subsubsection*{RG3}

\begin{itemize}
\item Cause serious/fatal disease in healthy adults, but can often be treated or fatality rate is low
\item One BSL-3 lab in Hyde Park (TB)
\item Examples: anthrax, bubonic plague, TB, high pathogenicity
\end{itemize}

\subsubsection*{RG4}

\begin{itemize}
\item Serious/fatal disease
\item Ebola, Marburg Virus, other hemorrhagic fever viruses
\end{itemize}

\subsubsection*{Important!}

\begin{itemize}
\item Do not dispose of chemicals and biologicals in the same container
\item Always wear appropriate PPE
\item Locate safety equipment in the lab
\item Remember to fill out UCAIR form if an accident happens after alerting TA
\item For chemical spills, clean with 70\% alcohol
\item For biological spills, decontaminate with bleach (pour over spill until the solution is 10\% bleach if the spill is less than 1 L)
\end{itemize}

\end{document}