\documentclass[letterpaper, 12pt]{article}
\usepackage{graphicx} % Required for inserting images
\usepackage{textcomp}
\usepackage{fullpage}
\usepackage{amsmath}
\usepackage{xcolor}
\usepackage{float}
\usepackage{geometry}
\usepackage{biblatex}
\geometry{margin=1in}
\usepackage{enumitem}
\usepackage{microtype}
\usepackage{gensymb}
\usepackage{parskip}
\usepackage{tikz}
\usepackage{caption}
\usepackage{cancel}
\usepackage{nicefrac}


\usepackage{hyperref}
\hypersetup{
  colorlinks=true,        % Enable colored links
  linkcolor=teal,         % Set color for internal links
  citecolor=teal,         % Set color for citations
  filecolor=teal,         % Set color for file links
  urlcolor=teal           % Set color for URLs
}

\usepackage[version=4]{mhchem}

\title{Carbohydrates}
\author{BIOS 10016}
\date{24 June 2025}

\begin{document}

\maketitle

\section*{Objectives}

\begin{itemize}
\item Know all definitions presented about Carbohydrates.
\item Classify sugars and describe relationships between sugars based on chemical and structural differences.
\item Use the nomenclature associated with sugar chirality and structure to name sugars in linear and hemiacetal/hemiketal forms.
\item Number the positions in a sugar.
\item Describe the differences between aldoses and ketoses.
\item Describe the properties of the functional groups in sugars, and how it enables them to interact with other molecules, like water.
\item Describe the formation, properties and prevalence of hemiacetal/hemiketal structures of sugars in solution.
\item Interconvert between the Fischer, Projection and Haworth representations.
\item Describe the reactions that sugars undergo:
\begin{itemize}
\item Difference between reducing and non-reducing sugars
\item Isomerization, esterification, and glycoside formation
\item Identify and describe deoxy and amino sugars
\end{itemize}
\item Describe the formation and naming of glycosidic linkages.
\item Identify and name the glycosidic linkages that can exist between sugars.
\item Describe the components, linkages and properties of disaccharides: lactose, maltose, and sucrose.
\item Describe the properties of branched and linear polysaccharides and their
dependence on the types of glycosidic linkages.
\begin{itemize}
\item Differences between starch, amylose, amylopectin and cellulose
\item Difference between amylose and heparin
\end{itemize}
\end{itemize}

\newpage

\section*{Carbohydrates}

\subsection*{Monosaccharides}
\begin{itemize}
\item Monosaccharides are the simplest carbohydrates and are aldehydes or ketones containing two or more hydroxyl groups
\item Aldehyde: aldoses; ketone: ketoses
\item Smallest monosaccharides are composed of 3 carbons
\item End carbon closest to the most oxidized carbon is 1
\item Often represented in \textbf{Fischer} or \textbf{Haworth}
\item Often named relative to glyceraldehyde: D, L (depends on where the hydroxyl group is)
\item D sugars are the most common in nature
\end{itemize}

\subsection*{Chirality}
D and L sugars are defined by the chiral carbon furthest from the most oxidized carbon and where the hydroxyl group is relative to that carbon.

3 carbons - \textbf{triose} \\
4 carbons - \textbf{tetrose} \\
5 carbons - \textbf{pentose} \\
6 carbons - \textbf{hexose}

\subsection*{Different forms of monosaccharides}

\begin{description}
\item [constitutional isomers] same formula, different attachment
\item [stereoisomers] same formula, different 3D orientation
\item [enantiomers] stereoisomers that are nonsuperimposable mirror images of each other (only look at the chiral carbons)
\item [diastereoisomers] stereoisomers that are not mirror images
\item [epimers] diastereoisomers that differ at only one asymmetric carbon atom
\item [anomers] diastereoisomers that differ at a new asymmetric carbon atom formed on ring closure (anomeric carbon (the carbon where the molecule attached differs): $\alpha$ is trans (\ce{CH2OH} and OH on opposite sides), $\beta$ is cis (\ce{CH2OH} and OH on the same side)
\end{description}

\textbf{Hemiacetal} or \textbf{hemiketal} sugars

Aldehyde + alcohol $\to$ hemiacetal \\
Ketone + alcohol $\to$ hemiketal

Happens through \textbf{nucleophilic attack}

$\alpha$ and $\beta$ designations based upon chirality at anomeric carbon

$\alpha$ and $\beta$ conformations are in equilibrium in water. The linear conformation is the least dominant species in solution

\subsection*{Naming of cyclic sugars}
Glucose forms a six-member ring: \textbf{pyranose} since it resembles pyran

Fructose forms a five-member ring: \textbf{furanose} since it resembles furan

$\alpha$ and $\beta$ anomers result from carbonyl containing functional groups having two faces.

\end{document}