\documentclass[letterpaper, 12pt]{article}
\usepackage{graphicx} % Required for inserting images
\usepackage{textcomp}
\usepackage{fullpage}
\usepackage{amsmath}
\usepackage{xcolor}
\usepackage{float}
\usepackage{geometry}
\usepackage{biblatex}
\geometry{margin=1in}
\usepackage{enumitem}
\usepackage{microtype}
\usepackage{gensymb}
\usepackage{parskip}
\usepackage{tikz}
\usepackage{caption}
\usepackage{cancel}
\usepackage{nicefrac}


\usepackage{hyperref}
\hypersetup{
  colorlinks=true,        % Enable colored links
  linkcolor=teal,         % Set color for internal links
  citecolor=teal,         % Set color for citations
  filecolor=teal,         % Set color for file links
  urlcolor=teal           % Set color for URLs
}

\usepackage[version=4]{mhchem}

\title{Enzyme kinetics}
\author{BIOS 10016}
\date{24 June 2025}

\begin{document}

\maketitle

\section*{Objectives}

\begin{itemize}
\item Describe the principles behind enzyme kinetics.
\item Describe the criteria and assumptions required to analyze a system using Michaelis-Menten kinetics, and the parameters that are yielded.
\item Prepare Michaelis-Menten and Lineweaver-Burk plots from raw data.
\item Calculate elements of the Michaelis-Menten equation when provided sufficient information.
\item Calculate the turnover number and specificity constant.
\item Extract kinetic data from the Lineweaver-Burk plot.
\item Describe the limits at which the turnover number and the specificity constant drive the reaction.
\end{itemize}

\newpage

\section*{Enzymes}

\subsection*{General properties of enzymes}

\begin{itemize}
\item Have high reaction rates (10$^6$ to 10$^{12}$ times faster than uncatalyzed reactions)
\item Mild reaction conditions $$2\ce{N2_{(g)} + 6\ce{H2(g)} \to \ce{2NH3(g)}}$$
\item Regulated activity
\item Reaction specificity
\begin{itemize}
\item Absolute specificity: only one substrate (urease $\to$ urea)
\item Relative specificity: few related substrates (e.g. hexokinase $\to$ glucose and fructose)
\item Stereospecific specificity: only one isomer (D or L form)
\end{itemize}
\end{itemize}

\subsection*{Active sites}
Substrates bind using noncovalent interactions to \textbf{active sites}. Active sites are a unique microenvironment that binds the substrate and catalyzes the reaction.

\textbf{Induced fit} = changing the shape of the enzyme to fit the substrate.

Initial binding of the substrate to the active site depends on IMFs.

\subsubsection*{Structural complementarity}

\textbf{Geometric}: substrate and enzyme structures match structurally

\textbf{Electronic}: substrate and enzyme structures match electronically (opposite charges)

\textbf{Stereospecificity}: chiralities match

\subsection*{Binding energy and transition states}

or $\Delta$G$_B$

The energy associated with substrate binding to the active site that pays for the entropy associated with binding.

Energy comes from:

\begin{enumerate}
\item Enthalpy from IMFs
\item Entropy of \ce{H2O}
\item Entropy of the substrate
\end{enumerate}

Transition states may block spontaneous processes. The transition state is short-lived, high-energy, and determines reaction rate.

To speed up: lower activation energy or increase $\Delta$G of S.

\textbf{Transition state analogs} are excellent enzyme inhibitors. The molecules mimic a transition state and bind to the active site, preventing substrate binding.

\textbf{Dissociation constant} - how tightly a substrate binds to a protein or enzyme. A smaller K$_D$ means stronger binding. (K$_i$ is K$_D$ for an inhibitor.)

\subsection*{Arrhenius equation}

\begin{equation}
k = Ae^{-\frac{E_a}{RT}}
\end{equation}

where $k$ is the rate constant, $A$ is the pre-exponential factor, $E_a$ is the activation energy, $R$ is the gas constant, and $T$ is the temperature in Kelvin.

\newpage

\section*{Kinetics}

\subsection*{Enzyme kinetics}

Consider a first-order reaction: an enzyme E converts one substrate S into one product P.

$$\text{S} \rightleftharpoons \text{P}$$

rate = $\displaystyle \frac{d[P]}{dt} = \frac{-d[S]}{dt}$

An enzyme is like a factory assembly line - rate of product formation depends on the time it takes to make a product (V$_{max}$, k$_{cat}$) and the time it takes for materials to arrive

$$ \text{E + S} \rightleftharpoons \text{ES} \rightleftharpoons \text{EP} \to \text{E + P}$$

$\text{E + S} \to \text{ES} $ = k$_1$ \\
$\text{ES} \to \text{E + S} $ = k$_-1$ \\
$\text{ES} \to \text{EP} $ = k$_2$ \\
$\text{EP} \to \text{ES} $ = k$_-2$

\subsection*{Dependence of reaction rate on [S]}

The more substrate, the greater the chance for collisions

rate = velocity = V $\propto$ [S]

To study enzyme kinetics, study it far away from equilibrium

v$_0$ = initial rate of reaction (initial velocities)

\subsection*{Reaction order}

The number of molecules that react in the rate determining step (slowest step)

\begin{description}
\item [first-order reaction] rate = k[S], rate $\propto$ [S]
\item [zero-order reaction] reaction rate is not proportional to [S], observed when [S] $>>>$ [E] on the flat line

rate = k$_{cat}$
\item [second-order reaction] reaction rate $\propto$ [S]$_1$ and [S]$_2$

$$ \text{pyruvate} + \ce{HCO3-} \to \text{oxaloacetate} $$

rate = k[S]$_1$[S]$_2$

\item [pseudo first-order reaction] reaction rate $\propto$ [S]$_1$ only (the independent substrate is present in excess amounts)

$$ \text{pyruvate} + \ce{HCO3-} \: \text{(excess)} \to \text{oxaloacetate} $$

rate = k[S]$_1$

\end{description}

\subsection*{Michaelis-Menten kinetics}

\begin{itemize}
\item Simplest, but very useful model
\item Requires collection of initial velocity data
\item Can be applied to study enzyme kinetics providing..
\begin{itemize}
\item 1st or pseudo 1st order
\item hyperbolic
\end{itemize}
\item Based upon the reaction $\text{E + S} \rightleftharpoons \text{ES} \to \text{E + P}$ or $\text{binding} \rightleftharpoons \text{processing}$
\item \textbf{Michaelis-Menten complex} = \textbf{K$_m$} = \textbf{enzyme-substrate complex} (ES)
\item Looking only at initial velocities (no k$_-2$)
\item Assumptions:
\begin{itemize}
\item Minimal amount of enzyme [E] $<<<$ [S]
\item No product ($<$10\%) and the forward reaction predominates, V = V$_0$
\item k$_{-1}$ $>>$ k$_2$ (ES $\to$ E + S is much faster than ES $\to$ E + P)
\item Steady state condition ([ES] will not change)
\end{itemize}
\end{itemize}

\subsubsection*{The Michaelis-Menten equation}

\begin{equation}
v_0 = \frac{V_{max}\text{[S]}}{\text{[S]} + K_m}
\end{equation}

where:

\begin{itemize}
\item $v_0$ = initial velocity
\item $V_{max}$ = maximum velocity, line approaches this value asymptotically
\item $[$S$]$ = substrate concentration
\item $K_m$ = Michaelis constant (the substrate concentration at which the reaction rate is half of $V_{max}$)
\end{itemize}

\subsubsection*{Extracting K$_m$ and V$_{max}$ values}
Linear plots:

\begin{itemize}
\item \textbf{The Lineweaver-Burke Double Reciprocal Plot}
\item Woolf-Hofstee Plot
\item Hanes-Woolf Plot
\item Eadie-Scatchrd Plot
\end{itemize}

Deriving the Lineweaver-Burk relationship:

$$v_0 = \frac{V_{max}\text{[S]}}{\text{[S]} + K_m}$$
$$ \frac{1}{v_0} = \frac{\text{[S]} + K_m}{V_{max}\text{[S]}} $$
$$ \frac{1}{v_0} = \frac{K_m}{V_{max}} \left(\frac{1}{[S]}\right) + \frac{1}{V_{max}}$$

Slope: $K_m$/$V_{max}$

x-int: $\displaystyle \frac{-1}{K_m}$

\subsubsection*{The Michaelis constant, K$_m$}

$$ K_m = \frac{k_{-1} + k_2}{k_1} $$

k$_2$ is negligible. Cancel and rearrange to get:

$$ K_m = \frac{k_{-1}}{k_1} = \frac{\text{[E]}\text{[S]}}{\text{[ES]}} $$

Units: M

Small value = tighter binding (high affinity)

Large value = weaker binding (low affinity) 

The constant value depends on 

\begin{itemize}
\item enzyme type
\item substrate
\item reaction conditions (pH, temperature, ionic strength)
\end{itemize}

K$_m$ is an approximation of substrate affinity, provided k$_2$ $<<<$ k$_{-1}$.

\subsubsection*{V$_{max}$}
A value that depends on enzyme properties

Has units: amount/time

\subsubsection*{Turnover number, k$_{cat}$, k$_2$}

$$k_{cat} = \frac{V_{max}}{[E]_T}$$

units: s$^{-1}$

\subsubsection*{The specificity constant}
A useful term to describe enzyme-substrate systems

Quantitative term to assess \textbf{catalytic efficiency} (how efficiently an enzyme econverts substrate to product

Specificity constant = $\displaystyle \frac{k_{cat}}{K_m}$ (rate of processing/binding of S)

Units: M$^{-1}$s$^{-1}$

$$\frac{k_{cat}}{K_m} = \frac{\text{higher rate}}{\text{higher affinity}} = \frac{k_{cat} \: \text{is bigger}}{K_m \: \text{is smaller}}$$

\subsubsection*{Dual nature of the Michaelis-Menten equation}

Combination of 0-order and 1st-order kinetics

The Michaelis Menten equation describes a hyperbolic dependence on [S]

Low-substrate concentration: $\displaystyle v = \frac{k_{cat}}{K_m}$

High-substrate concentration: $v = k_{cat}$

\end{document}
