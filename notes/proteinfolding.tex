\documentclass[letterpaper, 12pt]{article}
\usepackage{graphicx} % Required for inserting images
\usepackage{textcomp}
\usepackage{fullpage}
\usepackage{amsmath}
\usepackage{xcolor}
\usepackage{float}
\usepackage{geometry}
\usepackage{biblatex}
\geometry{margin=1in}
\usepackage{enumitem}
\usepackage{microtype}
\usepackage{gensymb}
\usepackage{parskip}
\usepackage{tikz}
\usepackage{caption}
\usepackage{cancel}
\usepackage{nicefrac}


\usepackage{hyperref}
\hypersetup{
  colorlinks=true,        % Enable colored links
  linkcolor=teal,         % Set color for internal links
  citecolor=teal,         % Set color for citations
  filecolor=teal,         % Set color for file links
  urlcolor=teal           % Set color for URLs
}

\usepackage[version=4]{mhchem}

\title{Protein structure and folding}
\author{BIOS 1006}
\date{20 June 2025}

\begin{document}

\maketitle

\section*{Objectives}

\begin{itemize}
\item Know all definitions.
\item Describe the reactions involving amino acids.
\item Describe properties of the bonds in the polypeptide backbone.
\item Describe and identify the different classes of protein structure: primary, secondary, tertiary and quaternary.
\item Describe the importance of primary structure in protein folding and the relationships between proteins.
\item Describe the properties of features, such as secondary structure elements and motifs, that are found in proteins.
\item Describe the forces that stabilize tertiary and quaternary structures of proteins.
\item Describe the properties of different protein types, classifications and architectures.
\item Understand the role of free energy, dynamics, the forces involved and the factors that influence, aid or impede protein folding, shape and function.
\end{itemize}

\newpage

\section*{Protein structure}

\subsection*{Peptide/amide bonds}

\textbf{Dehydration synthesis} or \textbf{condensation reaction}

Nitrogen with a lone pair attacks the carbonyl carbon of another amino acid, forming a covalent bond and releasing water. This requires energy and a \textbf{ribozyme} (enzyme made out of nucleotides, RNA) called a \textbf{ribosome} to catalyze the reaction.

\subsection*{Resonance structures}

Electrons (usually in double or triple bonds, or lone pairs) can move around
and be "shared'' or "delocalized'' over two or more atoms. This is called \textbf{resonance}.

\textbf{Resonance hybrids} result in the...

\begin{itemize}
\item \ce{C-N} bond having \textbf{partial double bond character}.
\item peptide bond being shorter and essentially planar. (\ce{sp^2} hybridized, trigonal planar)
\end{itemize}

Resonance structures have characteristics of both arrangements.

\subsubsection*{Resonance structures of the peptide bond}

C$\alpha$ on opposite sides of the amide bond = \textbf{trans} conformation (preferred form in proteins)

C$\alpha$ on the same side of the amide bond = \textbf{cis} conformation (less common, leads to steric clashes)

\subsection*{Dihedral angles}

\begin{itemize}
\item \textbf{$\phi$ (phi) angle}: nitrogen - $\alpha$ carbon, \ce{C-N-C_$\alpha$-C}
\item \textbf{$\psi$ (psi) angle}: $\alpha$ carbon - carbonyl carbon \ce{N-C_$\alpha$-C-N}
\item \textbf{$\omega$ (omega) angle}: carbonyl carbon - nitrogen \ce{C_$\alpha$-C-N-C_$\alpha$}
\item Both rotatable
\end{itemize}

\subsection*{Key terms for peptides}

\begin{description}
\item [residue] an amino acid within a peptide
\item [peptide] molecule containing amino acids linked together
\item [oligopeptide] molecule containing less than 10 amino acids
\item [polypeptide] molecule containing more than 10 amino acids
\item [proteins] functional molecules consisting of one or more polypeptides
\end{description}

\subsection*{Levels of protein structure}

\subsubsection*{Primary structure}
The unique sequence of amino acids that defines a peptide or polypeptide; shows all covalent bonds.

\subsubsection*{Secondary structure}
$\alpha$-helices (cylinders/arrows) and $\beta$-sheets (arrows, N- to C-terminus) are connected by random coils

\paragraph{The $\alpha$ helix} Looks like a spiral staircase, stabilized by hydrogen bonds between amide groups in the protein backbone. The hydrogen bonding pattern unique to alpha helix is that it requires hydrogen bonds between $i$ and $i+4$ (where $i$ is the amino acid of interest) and are stronger than the hydrogen bond between a regular amine and carboxyl group.

\begin{itemize}
\item 3.6 residues per turn
\item Pitch (distance between 2 identical points on adjacent turns): 5.4 Å (0.1 nm)
\item Rise (distance between 2 identical points on adjacent residues, pitch/residues per turn): 1.5 Å (0.15 nm)
\item Torsion angles: $\phi$ = -60$\degree$, $\psi$ = -45$\degree$
\item H-bonds parallel to helical axis and point in the same direction. The $\alpha$ helix has a
\item R-groups point
\item On average, $\sim$10 residues per helix
\end{itemize}

\paragraph{The $\beta$ sheet}

\end{document}