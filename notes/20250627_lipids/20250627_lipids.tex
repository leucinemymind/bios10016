\documentclass[letterpaper, 12pt]{article}
\usepackage{graphicx} % Required for inserting images
\usepackage{textcomp}
\usepackage{fullpage}
\usepackage{amsmath}
\usepackage{xcolor}
\usepackage{float}
\usepackage{geometry}
\usepackage{biblatex}
\geometry{margin=1in}
\usepackage{enumitem}
\usepackage{microtype}
\usepackage{gensymb}
\usepackage{parskip}
\usepackage{tikz}
\usepackage{caption}
\usepackage{cancel}
\usepackage{nicefrac}


\usepackage{hyperref}
\hypersetup{
    colorlinks=true,        % Enable colored links
    linkcolor=teal,         % Set color for internal links
    citecolor=teal,         % Set color for citations
    filecolor=teal,         % Set color for file links
    urlcolor=teal           % Set color for URLs
}

\usepackage[version=4]{mhchem}

\title{Lipids and membranes}
\author{BIOS 1006}
\date{27 Jun 2025}

\begin{document}

\maketitle

\section*{Objectives}

\begin{itemize}
\item Know all definitions.
\item Describe the roles played by lipids.
\item Describe micelle and bilayer structures.
\item Describe the forces that stabilize lipid structures.
\item Predict the relative melting points of fatty acids and the lipids in which they are found.
\item Draw a fatty acid given a name or designation: (e.g. trans-18:1$^{\Delta 9}$ or 18:1$\omega$9) and name fatty acids if given a structure.
\item Describe the essential and non-essential designations.
\item Recognize a triacylglycerol or phospholipid given the names, structures or designations of the components.
\item Describe the relationship between lipid fluidity and melting point.
\item  If given a list of fatty acids or lipids, rank the melting points.
\item  Describe the biological roles, the chemical reactions and the physical and structural properties of each lipid class.
\item  Determine the class of a lipid if given a structure.
\item Determine the class of isoprenoids, and locate isoprene units in molecules.
\end{itemize}

\newpage

\section*{Lipids}
Are water-insoluble molecules that are highly soluble in organic solvents. They are defined by their solubility (in organic solvents, not water)

\subsection*{Roles of lipids}

\begin{itemize}
\item Structure (membranes)
\item Energy storage (fats, oils)
\item Protection (antioxidants, water-proofing)
\item Pigments (carotenoids)
\item Coenzymes (vitamins, heme)
\item Signaling (hormones, growth factors)
\end{itemize}

\subsubsection*{Lipids do not form covalent polymers!}
Recall:

\begin{itemize}
\item proteins are covalent polymers of amino acids
\item nucleic acids are covalent polymers of nucleotides
\item polysaccharides are covalent polymers of sugars
\end{itemize}

Lipids also form higher order structures, but the esubunits are not covalently attached. These structures are micelles and bilayers (membranes) that are maintained by IMFs such as LDFs (non-polar parts), hydrogen bonds, dipole-dipole forces, and ion-dipole forces (between lipids and \ce{H2O}).

\subsection*{Classes and nomenclature of lipids}

\begin{itemize}
\item Free fatty acids
\item Triacylglycerols
\item Phospholipids
\item Glycolipids
\item Isoprenoids
\end{itemize}

\subsubsection*{Fatty acids}
Monocarboxylic acids composed of a long hydrocarbon chain with a carboxyl group at the end

\begin{itemize}
\item Even number of carbon atoms
\item Generally unbranched chain (straight chain)
\item Hydrophilic and hydrophobic ends (amphipathic)
\item Varying degrees of saturation
\item Produced by organisms and synthesized from Acetyl-CoA
\item Plants make all the fatty acids they need
\item We only produce some fatty acids, the rest must be obtained from our diet (essential fatty acids)
\item Essential: in your diet
\item Non-essential: can be synthesized by your body (doesn't mean not important!)
\item We need to consume \textbf{essential fatty acids} (omega-6, omega-3)
\item Think of the numbering scheme from the $\omega$-carbon
\item Saturated fatty acids are flexible (free rotation aroudn C-C bonds)
\item Linear conformation ist he most stable due to steric constraints
\item Chains pack tightly against each other and form more rigid, organized aggregates (i.e. membranes)
\item London Force Strength is proportional to surface area, depends on length and proximity
\end{itemize}

\begin{description}
\item [saturated fatty acids] 0 double bonds
\item [monounsaturated fatty acids] 1 double bond
\item [polyunsaturated fatty acids] 2 or more double bonds
\end{description}

Double bonds can result in one of two orientations. (cis (favored)/trans)

\subsubsection*{Saturated fatty acid nomenclature} 

\begin{itemize}
\item 18 carbon saturated fatty acid: 18:0 (carbons:double bonds)
\item IUPAC: Octadecanoate (protonated: octadecanoic acid)
\item Common: Stearate (protonated: stearic acid)
\end{itemize}

14:0 \textbf{tetra}\underline{deca}noic acid (tetra = 4, deca = 10) \\
16:0 hexadecanoic acid \\
18:0 octadecanoic acid \\
20:0 \textbf{eicosanoic} acid \\
22:0 \textbf{do}\underline{cos}anoic acid (do = 2, cos = 22)\\
24:0 tetracosanoic acid \\
26:0 hexacosanoic acid

\subsubsection*{Monounsaturated fatty acid nomenclature}

18 carbon cis monounsaturated fatty acid with a double bond \textbf{starting} at position 11

\begin{itemize}
\item Full IUPAC name: cis $\Delta^{11}$-octadecene/-oic acid
\item IUPAC shorthand: 18:1$^{\Delta 11}$
\item 18:1 $\omega$ 7 (counting backwards)
\end{itemize}

\subsubsection*{Polyunsaturated fatty acid nomenclature}

18 carbon cis monounsaturated fatty acid with 2 double bonds \textbf{starting} at positions 9 and 12
\begin{itemize}
\item Full IUPAC name: cis-cis- $\Delta^{9}$, $\Delta^{12}$-octadecadienoic acid (dien = 2 double bonds)
\item IUPAC shorthand: 18:1$^{\Delta 9, \Delta 12}$
\item 18:2 $\omega$ 6 (from the first double bond counting backwards. assuming all are cis and any other double bond is 3 carbons away)
\end{itemize}

\subsection*{Lipid fluidity}
The higher the melting point, the lower the fluidity (this is why butter is solid and olive oil is liquid)

Also depends on chain length. Longer chains have...

\begin{itemize}
\item Greater surface area
\item Stronger London Forces
\item Increased melting point
\end{itemize}

Lipid fluidity also depends on degree of unsaturation. \textbf{More double bonds = lower melting points} (kinks in structure preventing close interactions)

Trans fats (human-introduced by hydrogenation, more solid, higher melting point) are more easily packed together than cis fats (more liquid, lower melting point).

\subsection*{Triacylglycerols}
\begin{itemize}
\item Glycerol backbone (linear vertical chain of CH-OH)
\item Ester linkage containing 3 fatty acid chains and alcohol
\item Constituent fatty acid lengths and degrees of unsaturation can vary (fats or oils), depending on number of double bonds/carbons
\item Primarily used for energy storage in animal (adipose) and plants
\item \textbf{Simple} triacylglycerols: All fatty acids are the same
\item \textbf{Mixed} triacylglycerols: Fatty acids are different
\end{itemize}

\subsection*{Phospholipids}
\begin{itemize}
\item Involved in generation of signaling molecules, anchoring proteins, and membrane formation
\item An amphipathic derivative of glycerol (3 OH groups) or sphingosine
\end{itemize}

\subsubsection*{Phosphoglycerides (phospholipids)}

\begin{itemize}
\item Composed of glycerol, two fatty acids (R groups), phosphate group, and ``X'' (amino alcohol or hydrogen)
\item Simplest form: phosphatidic acid (X=H) and R = fatty acids
\item Others classified according to the amino alcohol attached (X)
\item Fatty acids attached can be the smae or different
\item Example: \textbf{lecithin} (phosphatidylcholine), pH insensitive, a phosphoglyceride with choline
\item \textbf{Cephalins} (phosphoglycerides with ethanolamine or serine) such as phosphatidylserine and phosphatidylethanolamine
\end{itemize}

\subsubsection*{Sphingolipids (includes phospholipids)}

\begin{itemize}
\item Hydrophobic tail
\item Ignored OH group
\item Amide with fatty acid
\item OH group that can attach a sugar (acetal) or a phosphate-amino alcohol
\item Sphingosine + fatty acid = seramide
\item General structure: sphingosine (sphingolipid) + fatty acid
\end{itemize}

\subsubsection*{Spingophospholipids}
Sphingosine + fatty acid + phosphate + amino alcohol (e.g. A sphingomyelin, brain and nervous system tissue)

\textbf{Find the amide bond!}

\subsubsection*{Sphingoglycolipids}
Sphingosine + fatty acid + carbohydrate

\subsubsection*{Properties of phosphoglycerides and sphingolipids}
Hydrophilic head and hydrophobic tails

\subsection*{Isoprenoids}
Generated from acetyl-coA, composed of isoprene subunits (CH2-C=CH-CH2-)$_n$

2 classes: \textbf{terpenes} and \textbf{steroids}

\subsubsection*{Terpenes}
Large and diverse class of strong-smelling organic compounds, mostly produced by plants.

Classified by the number of isoprene or terpene units (\textbf{1 terpene unit = 2 isoprene units})

\begin{description}
\item [monoterpenes] 2 isoprene units, used in perfumes
\item [sesquiterpenes] 3 isoprene units, citronella
\item [tetraterpenes] 8 isoprene units, carotenoids
\item [polyterpene] 1000s of isoprene units, rubber
\end{description}

\textbf{Mixed terpenoids} - also contain a non-isoprene component (e.g. ubiquinone, vitamin K)

\subsubsection*{Steroids}
Triterpene derivatives (6 isoprene units, 4 fused rings)

\textbf{Cholesterol:}
\begin{itemize}
\item Basic structure with multiple rings and alcohol (hydroxyl) group at the end. 4 fused carbon rings, 2 stacked on top of the other 2, mismatched
\item Found in all eukaryotes and some bacteria
\item A component of plasma membranes
\item Precursor for all steroid hormones, Vitamin D and bile salts
\item Found as esters with fatty acids and sugars
\end{itemize}

\subsection*{Micelles and membranes}

\subsubsection*{Structures of lipid assemblies}
\textbf{Conical} lipids such as fatty acids form micelles (spherical, fatty acids, detergents)

\textbf{Cylindrical} lipids such as glycerophospholipids form membranes (rectangular, packed closely)

\textbf{Inverted conical} lipids such as triacylglycerols also form membranes 

\subsubsection*{Lipid bilayers}
Phospholipids and glycolipids prefer to form bilayer structures in aqueous solutions.

\begin{itemize}
\item Structure: Sheet-like
\item Thickness: 2 leaflets (outer + inner), 30-40Å
\item Stabilized by IMFs (tails: IMFs, heads: with \ce{H2O} - ion-dipole, dip-dip, H-bond)
\item ``Fluid Mosaic Model'' (Singer \& Nicholson)
\item Membranes are dynamic structures composed of proteins and phospholipids
\item The bilayer is a fluid matrix (lateral diffusion)
\item Lipids and proteins can rotate and freely migrate or diffuse within a leaflet
\end{itemize}

\end{document}