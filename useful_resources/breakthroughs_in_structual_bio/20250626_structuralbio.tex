\documentclass[letterpaper, 12pt]{article}
\usepackage{graphicx} % Required for inserting images
\usepackage{textcomp}
\usepackage{fullpage}
\usepackage{amsmath}
\usepackage{xcolor}
\usepackage{float}
\usepackage{geometry}
\usepackage{biblatex}
\geometry{margin=1in}
\usepackage{enumitem}
\usepackage{microtype}
\usepackage{gensymb}
\usepackage{parskip}
\usepackage{tikz}
\usepackage{caption}
\usepackage{cancel}
\usepackage{nicefrac}


\usepackage{hyperref}
\hypersetup{
  colorlinks=true,        % Enable colored links
  linkcolor=teal,         % Set color for internal links
  citecolor=teal,         % Set color for citations
  filecolor=teal,         % Set color for file links
  urlcolor=teal           % Set color for URLs
}

\usepackage[version=4]{mhchem}

\title{History and recent breakthroughs in structural biology}
\author{BIOS 10016}
\date{26 June 2025}

\begin{document}

\maketitle

\subsection*{Architecture vs. structural biology}

\begin{itemize}
\item Structural biology lets us go inside molecules and see how they are built so we can understand how life functions at the smallest scale
\item You can't get any information about how the city functions out of looking at the names of buildings on a map - you have to go inside the buildings and see how they're built
\item Structural similarities between buildings and molecules as part of a larger structure
\item Architects study buildings; biologists study biomolecules
\item Blueprint: architectural blueprint/bricks, beams, steel, concrete/DNA
\item Building blocks: walls, floors, support beams/amino acids, nucleotides...
\item Structural elements: alpha helices, beta sheets, domains...
\item Methods: models and simulations/X-ray, cryo-EM, NMR...
\item Structural failure: collapse, poor design/aggregation, diseases...
\item Renovation/engineering: repurposing old buildings/protein engineering, drug design...
\end{itemize}

\subsection*{Studying the structure of molecules}
\begin{itemize}
\item First Nobel Prize in physics - Wilhelm Conrad Röntgen, 1901, discovered X-ray
\item X-rays are electromagnetic waves with a higher frequency than visible light
\item ``Seeing'' the structure: laser goes through a sieve at a scattering angle (diffraction pattern)
\item Sieves can diffract lasers because they have very tiny meshes
\item Diffraction patterns reveal underlying structures
\item 1914: Max von Laue was awarded the Nobel Prize in Physics ``for his discovery of the diffraction of X-rays by crystals'' (mineral crystals, not protein crystals)
\item 1915: Sir William Henry Bragg and William Lawrence Bragg (proposed the equation at 22, youngest Nobel Prize awarded to a scientist at only 25!) were awarded the Nobel Prize in Physics ``for their services to the analysis of crystal structure by means of X-rays'' (again, still minerals)
\item Bragg's Law: $2d\sin\theta = \lambda$ (same pattern, but wavelengths of colors differ - red pattern appears larger than green)
\end{itemize}

\subsection*{X-ray crystallography}
\begin{itemize}
\item People noticed that proteins can be crystallized too! (e.g. haemoglobin crystals)
\item In earlier days, X-ray crystallography focused on minerals and salts, not proteins
\item Breakthrough by J.D. Bernal and Dorothy Crowfoot (Hodgkin) in 1934, who crystallized pepsin and determined its structure
\item Bernal discovered that protein crystals must be kept hydrated to produce meaningful diffraction patterns
\item When exposed to air, birefringence diminishes (crystal structure breaks)
\end{itemize}

\subsection*{Breakthrough on studying fiber structures}
\begin{itemize}
\item One-dimensional crystals (linear) e.g. $\alpha$-keratin, collagen
\item Two-dimensional crystals (planar) e.g. silk fibroin, $\beta$-keratin
\item Different diffraction patterns for different structures
\item 1954: Linus Pauling was awarded the Nobel Prize in Chemistry ``for his research into the nature of the chemical bond and its application to the elucidation of the structure of complex substances''
\item But he drew a \textit{left-handed $\alpha$ helix} (gasp)
\item This was before people knew that $\alpha$ helices tended to show up in their right-handed form
\item Photograph 51: diffraction of DNA by Rosalind Franklin, Francis Crick, and James Watson
\item 1962: James Watson, Francis Crick, and Maurice Wilkins (Franklin's mentor) were awarded the Nobel Prize in Physiology or Medicine ``for their discoveries concerning the molecular structure of nucleic acids and its significance for information transfer in living material''
\item 1962: Max Perutz and John Cowdery Kendrew were awarded the Nobel Prize in Chemistry ``for their studies of the structures of globular proteins''
\item 1964: Dorothy Crowfoot Hodgkin was awarded the Nobel Prize in Chemistry ``for her determinations by X-ray techniques of the structures of important biochemical substances'' (insulin!)
\end{itemize}

\subsection*{Using electrons as an alternative method to probe structure}
\begin{itemize}
\item 1906: The Nobel Prize in Physics was awarded to J. J. Thomson ``in recognition of the great merits of his theoretical and experimental investigations on the conduction of electricity by gases''
\item 1929: The Nobel Prize in Physics was awarded to Louis de Broglie ``for his discovery of the wave nature of electrons''
\item 1937: The Nobel Prize in Physics was awarded to Clinton Joseph Davisson and George Paget Thomson ``for their experimental discovery of the diffraction of electrons by crystals''
\item 1986: The Nobel Prize in Physics was awarded to Ernst Ruska, Gerd Binnig, and Heinrich Rohrer ``for their design of the first electron microscope'' (Ruska) and ``for their design of the scanning tunneling microscope'' (Binnig and Rohrer)
\item 1974: The Nobel Prize in Physiology or Medicine was awarded jointly to Albert Claude, Christian de Duve and George E. Palade ``for their discoveries concerning the structural and functional organization of the cell''
\end{itemize}

\subsubsection*{cryoEM}
Freezing molecules, taking an electron micrograph, boxing out individual chaperone proteins, and averaging them to get 3D structure

It took decades for cryoEM to reach the point where it could be used to determine the structure of proteins at high resolution

2017: The Nobel Prize in Chemistry was awarded jointly to Jacques Dubochet, Joachim Frank, and Richard Henderson ``for developing cryo-electron microscopy for the high-resolution structure determination of biomolecules in solution''

Easier determination of protein structure - ``resolution revolution of electron microscopy''!!

\subsection*{Modern advancements}

\begin{itemize}
\item AlphaFold!!!!!!!!!!!
\item Training data of high quality - algorithm is only as good as the training data
\item 2024: The Nobel Prize in Chemistry was divided, one half awarded to David Baker ``for computational protein design'' and the other half jointy to Demis Hassabis and John Jumper ``for protein structure prediction''
\item \textbf{What are the remaining challenges?}
\begin{itemize}
\item Structure determines function, but knowing the structure does not equal knowing the function.
\item Structures are static, but molecules are moving!
\item Intrinsically disordered protein (IDP)
\item New methods need to be developed to see a molecular movie instead of static pictures
\end{itemize}
\item Molecular structures are not just cool - they're essential! They give us a powerful way to understand how life works at the deepest level, revealing how nature has shaped these molecules over millions of years to make us who we are.
\end{itemize}

\end{document}